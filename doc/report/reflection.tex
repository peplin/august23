\chapter{Reflection}
The members of the group individually reflected on some of the results of the project, technically, socially and otherwise.  Rather than condense them into a single voice, we include them here separately to showcase the overarching individuality among the group. 

\paragraph{Jiangang Hao}

\paragraph{Brian Nord}

\paragraph{Christopher Peplin}
The most important goal to me, and one we did accomplish, was to complete a vertical slice of an entire system. Our gallery installation and the hardware and software that support it touch an amazing number of topics, and I know everyone on the team has learned a great deal about their own and one another's fields in the process. I am extremely happy to have expanded my knowledge with hands on experience in:
\begin{itemize}
\item XML-RPC
\item Java Servlets
\item Multi-threading and databases in Java
\item Sound in the Processing environment
\item Library development for Processing
\item Releasing a software project online with an open source license
\item Multitouch input processing
\item Graphics performance optimization
\item Analog signal processing (filters, opamps)
\end{itemize}
I am coming away from this semester having created multiple tools, and I am eager to pass those on to anyone else with an interest in these topics. 


Our team used an online wiki (a TiddlyWiki) \cite{WIKI} to organize our thoughts and coordinate the project plan. I was very pleased with how this fit into the group's workflow. The wiki served as the default place to put any thoughts or meeting notes, which greatly simplified the process of sharing information.

Despite being well-organized, the project lost focus around the middle of the semester. I think the primary cause was the independent spirit of the group  members. Each person's perception of the group direction was slightly different and ultimately led the team too far from any core mission. The end of the project was extremely rushed as a result. Our mission changed significantly (perhaps too much) after the first design review, and we lost some ground when some completed work was no longer as relevant. During this period, I continued work on the Twoverse backed, which ultimately played a strong role in the software system for the gallery. With better planning, we would have started work on the gallery display software sooner, giving us more time to test the entire system. 

Overall, I enjoyed working with people outside of the computer science department. When I work with people with similar types of knowledge, I found that I began to make assumptions about ways of thinking and ways of looking at problems. This project made me realize how the "common sense" answer to me is often more software oriented not as obvious to a sculptor or physicist, who may have a more materials or theoretical solution. Accepting the existence of other solutions is sometimes a challenge, but ultimately a rewarding task.

\paragraph{John Walters}
My personal response to this project is one that highlights my great appreciation for the manner in which all members contributed to the final conception of the exhibition, even when it seemed that we may never see the project come to fruition. It was the delegation of tasks within a group dynamic that allowed each member to contribute significantly. By using each person’s skills to their fullest potential, while at the same time being aware of the overall outcome, everyone was able to apply themselves as well as learn from this collaborative experience.    

