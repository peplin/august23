\section{Multitouch Table}
The multitouch table created for the project is a rear
diffused
illumination
design (see figure \ref{fig:mtdiagram}), and is housed in a
core‐ten
steel
cabinet with
recessed
cooling
fans
and
an access
panel
on
the
rear
vertical
wall.

The
touch
surface
is
a
1⁄2"
polycarbonate
sheet with
an
adhesive
projection
film
applied
to
the
underside (acting as a diffuser for the projected image).



In
addition,
infrared
light
is
also
projected at
the
diffuser
from
below (inside the cabinet)
the
touch
surface. The table currently uses an array of six multiple IR LED lamps.
When
an
object
touches
the
surface
it
reflects
more
light
than
the
diffuser
or
objects
far away from the surface.
The change in light is
detected by a web cam placed inside the cabinet, and the signal is fed to the computer for analysis.

\begin{figure}[htp]\centering
  \includegraphics[width=.8\textwidth]{images/mt-diagram.png}
  \caption{This is a multitouch table!}\label{fig:mtdiagram}
\end{figure}
\subsection{Software}
The multitouch table uses The Beta, from the NUI Group \cite{NUI}, to process the video stream from the webcam. The Beta, tbeta for short, is an open source tool that analyzes a video to find tracking data for objects it recognizes as fingers or cursor devices. The software provides a great deal of control over the video parameters (high-pass filter, amplification, threshold, etc.) that adapts well to many types of multitouch displays.

Tbeta outputs the tracking data using the TUIO protocol \cite{TUIO}, which is an open framework for receiving input events in various programming environments. For this project, the TUIO events sent by tbeta were received using the open source Java TUIO library in a Processing sketch. 
